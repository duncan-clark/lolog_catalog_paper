% Options for packages loaded elsewhere
\PassOptionsToPackage{unicode}{hyperref}
\PassOptionsToPackage{hyphens}{url}
%
\documentclass[
]{statsoc}
\usepackage{lmodern}
\usepackage{amssymb,amsmath}
\usepackage{ifxetex,ifluatex}
\ifnum 0\ifxetex 1\fi\ifluatex 1\fi=0 % if pdftex
  \usepackage[T1]{fontenc}
  \usepackage[utf8]{inputenc}
  \usepackage{textcomp} % provide euro and other symbols
\else % if luatex or xetex
  \usepackage{unicode-math}
  \defaultfontfeatures{Scale=MatchLowercase}
  \defaultfontfeatures[\rmfamily]{Ligatures=TeX,Scale=1}
\fi
% Use upquote if available, for straight quotes in verbatim environments
\IfFileExists{upquote.sty}{\usepackage{upquote}}{}
\IfFileExists{microtype.sty}{% use microtype if available
  \usepackage[]{microtype}
  \UseMicrotypeSet[protrusion]{basicmath} % disable protrusion for tt fonts
}{}
\makeatletter
\@ifundefined{KOMAClassName}{% if non-KOMA class
  \IfFileExists{parskip.sty}{%
    \usepackage{parskip}
  }{% else
    \setlength{\parindent}{0pt}
    \setlength{\parskip}{6pt plus 2pt minus 1pt}}
}{% if KOMA class
  \KOMAoptions{parskip=half}}
\makeatother
\usepackage{xcolor}
\IfFileExists{xurl.sty}{\usepackage{xurl}}{} % add URL line breaks if available
\IfFileExists{bookmark.sty}{\usepackage{bookmark}}{\usepackage{hyperref}}
\hypersetup{
  pdftitle={Supplement: ``Comparing the Real-World Performance of Exponential-family Random Graph Models and Latent Order Logistic Models''},
  hidelinks,
  pdfcreator={LaTeX via pandoc}}
\urlstyle{same} % disable monospaced font for URLs
\usepackage{graphicx,grffile}
\makeatletter
\def\maxwidth{\ifdim\Gin@nat@width>\linewidth\linewidth\else\Gin@nat@width\fi}
\def\maxheight{\ifdim\Gin@nat@height>\textheight\textheight\else\Gin@nat@height\fi}
\makeatother
% Scale images if necessary, so that they will not overflow the page
% margins by default, and it is still possible to overwrite the defaults
% using explicit options in \includegraphics[width, height, ...]{}
\setkeys{Gin}{width=\maxwidth,height=\maxheight,keepaspectratio}
% Set default figure placement to htbp
\makeatletter
\def\fps@figure{htbp}
\makeatother
\setlength{\emergencystretch}{3em} % prevent overfull lines
\providecommand{\tightlist}{%
  \setlength{\itemsep}{0pt}\setlength{\parskip}{0pt}}
\setcounter{secnumdepth}{-\maxdimen} % remove section numbering
\usepackage{amsmath}\usepackage{amsfonts}\usepackage{amssymb}\usepackage{mathrsfs}\usepackage{graphicx}\usepackage{float}\usepackage{url}\usepackage{epstopdf}\usepackage{booktabs}\usepackage{longtable}\usepackage{appendix}\usepackage[a4paper]{geometry}\usepackage{graphicx}\usepackage[textwidth=8em,textsize=small]{todonotes}\usepackage{amsmath}\usepackage{natbib}\usepackage{hyperref}\usepackage{xcolor}\usepackage{dsfont}\usepackage{colortbl}
\usepackage{booktabs}
\usepackage{longtable}
\usepackage{array}
\usepackage{multirow}
\usepackage{wrapfig}
\usepackage{float}
\usepackage{colortbl}
\usepackage{pdflscape}
\usepackage{tabu}
\usepackage{threeparttable}
\usepackage{threeparttablex}
\usepackage[normalem]{ulem}
\usepackage{makecell}
\usepackage{xcolor}

\title{Supplement: ``Comparing the Real-World Performance of Exponential-family
Random Graph Models and Latent Order Logistic Models''}
\author{}
\date{\vspace{-2.5em}}

\begin{document}
\maketitle

\newcommand{\R}{\mathbb{R}}
\newcommand{\N}{\mathbb{N}}
\newcommand{\E}{\mathbb{E}}
\newcommand{\V}{\mathbb{V}}
\newcommand{\bfR}{\mathbf{R}}
\newcommand{\bfX}{\mathbf{X}}
\newcommand{\bfW}{\mathbf{W}}
\newcommand{\bfD}{\mathbf{D}}
\newcommand{\INT}{\int_{-\infty}^{+\infty}}
\newcommand{\p}{\partial}
\newcommand{\ra}{\Rightarrow}
\newcommand{\dH}{d\mathscr{H}}
\newcommand{\ch}{\text{cosh}}
\newcommand{\sh}{\text{sinh}}
\newcommand{\ex}{\mathbb{E}\left[X\right]}
\newcommand{\ey}{\mathbb{E}\left[Y\right]}
\newcommand{\logit}{{\rm logit}}
\newcommand{\MOM}{{\rm MOM}}

\setcounter{secnumdepth}{4}

\section{Individual Network Modelling Comments}\label{app:comments}

This section contains descriptions of each of the individual networks in
the ensemble of networks considered in ``Comparing the Real-World
Performance of Exponential-family Random Graph Models and Latent Order
Logistic Models''

\subsection{Add Health}

This was a network of high school students, obtained from the well
studied National Longitudinal Study of Adolescent to Adult Health
\citep{AddHealth2007}. Networks from the survey have been fit using
ERGMs {[}\cite{Goodreau2007}, \cite{Hunter_Goodreau_2008}{]}. There are
multiple networks available but the particular network in this case has
1681 adolescents/nodes with covariates for grade, gender and race
provided.

Analysis of the network with ERGMs, yields insights into the typical
relationships between nodes of different and common grade, gender and
race. Notably the tendency towards homophily within all grades as well
as between White and Black students but not Hispanic students. The
models allow for the strong interdependence of network ties, using the
GWESP term.

We were able to fit ERGMs and LOLOG with the published ERGM terms but
the models did not fit the data well, as noted extensively in
\cite{Goodreau2007}.

\cite{Goodreau2007} provided extensive commentary on the goodness of fit
of many ERGM models, the authors considered the degree and ESP
distributions as well as the distribution of geodesic distances between
people. A good fit on the degree distribution was only able to be
achieved by the authors by including terms that sacrificed the fit on
the ESP distribution. The LOLOG models exhibited similar problems,
however we were able to fit a LOLOG model with triangles and stars to
achieve an improved fit, but did not eliminate this issue.

\subsection{Junior High}

These data are 102 friendship networks in junior high school.
\cite{Lubbers2007} performed reanalysis of 102 networks consider
pseudo-likelihood and the then recent MCMC-MLE methods. We omitted this
from our study due to its size, and the fact that is atypical of the
usual applied social network analyses that ERGM is used for.

\subsection{Kapferer's Tailors}

The paper that fit this ERGM was \cite{Robins2007}. The authors in this
paper were investigating applying novel specifications to a range of
networks available through the UCINET software. We note that the models
were fitted using \texttt{pnet} software.

In the Kapferer Tailor Shop networks, the nodes are workers in a Zambian
tailor shop, with two different interactions, social and
``instrumental'' (work or assistance related). These were collected at
two distinct time points giving 4 networks. \cite{Robins2007} stated the
ERGM fit of the kapfts1 - the first social interaction network, of 39
workers, which is what concerns us here.

Briefly the qualitative conclusion of the analysis in \cite{Robins2007}
was that the network exhibited a tendency towards dense regions of
overlapping triangulation within a core periphery structure.This was
suggested by the significant and positive GWESP parameter in addition
the lack of significant alternating k star parameter. The authors
comment that this suggests the network exhibits a small number of
popular tailors, with the underlying social interaction network driven
by social transitivity.

Our estimated coefficients were different to those stated by the
authors, though the results are not qualitatively different, the use of
\texttt{pnet} instead of the \texttt{ergm} package may contribute to
this. We were unable to recreate the fit when including a 2-star
parameter unlike the authors who state a result for this. We were able
to fit an ERGM with high decay parameters, but this neither matched the
published ERGM, nor provided any extra insight. We were also unable to
fit the network to an ERGM with triangle and star parameters as stated
by the authors.

We were able to fit LOLOG models to the network using the geometrically
weighted terms. However in contrast to ERGM we were able to fit LOLOG
just with triangle, 2 and 3 star parameters.

Using higher order terms as in the published fit, the fit of the LOLOG
and ERGM models were poor on the degree and ESP distribution of the
network. Fitting the LOLOG model with triangle and star terms was a
slight improvement.

The qualitative conclusion of the authors was was consistent with the
LOLOG triangle and star model interpretation.

\subsection{Florentine Families}

This network was the second network fit in \cite{Robins2007}. In this
network the nodes are 16 influential families in Florence in the 1500s.
Marital networks and business tie networks are available with the fit
published being the business network.

The published fitting focused on structural terms, including nodal
covariates did not have a large affect on the coefficients. The
qualitative conclusion of the analysis was a high level of social
transitivity, and a tendency for non-isolated families to have multiple
business ties with other families, but with a ceiling on the likely
number of such ties. This likely reflects the difficulty of a family
maintaining business ties with an increasing number of different
families, in a commercially competitive environment.

We were able to recreate the published ERGM and fit LOLOG model with the
same terms. Both models fitted the observed network well. However the
LOLOG model parameters had high estimated variance, suggesting that the
model fit could be sensitive to variation in the data. This limits the
interpretation possible from the LOLOG model. We do however note that
variance estimates for both the LOLOG and ERGM model are only
asymptotically valid, so are likely not valid for such a small network.

\subsection{German Schoolboys}

This network is a directed network of friendships between German
schoolboys in class from 1880 to 1881, collected by Johannes Delitsch,
in one of the earliest studies to engage a network based approach. This
was reanalysed in \cite{Heidler2014} with ERGMs, and compared with
similar friendship networks in schools today.

Nodal covariates available were academic class rank, whether the student
was repeating class rank, whether the student gave sweets out, and
whether the student was handicapped or not. Note that academic rank also
has a spatial component since the schoolboys were sat in order of their
academic rank in the classroom.

The authors concluded that the pupils network had a tendency towards
reciprocated friendships. The also concluded the triadic closure
observed is generated through transitivity and not through generalized
exchange, due to the lack of significant cycling triple parameter when
included. In addition the analysis concluded high academic class rank
students were more likely to have more friendships and that friendship
nominations tended to be hierarchical. That is pupils tended to
nominated other pupils ranked higher than themselves as friends. There
was also interest in the four repeaters and the `sweets giver' have who
were concluded to have disproportionately high popularity even after
allowing for the other social structure of the network. The opposite was
concluded for impaired pupils.

We were able to match the models in the paper, which used a wide array
of network terms. We found models fitted using star and triangle
parameter to be degenerate. We noted that the models in the paper did
not include GWDEG terms as is usual to account for social popularity
processes.

We were not able to fit LOLOG models using the terms in the published
ERGM fit. However substituting the geometrically weighted ESP term for
triangle term allowed for the fitting of the LOLOG model. The published
ERGM and LOLOG model with triangle term substituted both fit the
observed network well.

The LOLOG model interpretation was broadly consistent with the ERGM
interpretation with some small differences on various nodal covariate
terms.

We also experimented with constraining the orderings by nodal covariates
for this network. Introducing rank based ordering i.e.~considering edges
involving higher ranked boys first (least academically able) increases
the up-rank effect and produces a highly significant nodal rank effect.
As we are considering high rank boys first, in the generating process,
if all else were equal they become ``filled up'', i.e.~highly connected
before the lower rank boys are added. However we observe in the data low
rank boys nominating high rank boys as friends. To counteract the
negative effect on tie formation between low and the ``filled up'' high
rank boys, the up-rank effect increases. This impresses upon us the need
to interpret LOLOG fits conditional on the specified ordering process,
in particular when the ordering process is based on nodal covariates.

\subsection{Employee Voice}

This data set contained 6 directed networks of between 24 and 39 nodes
of employee voice, i.e.~making a suggestion or voicing a problem from a
speaker to a recipient \cite{Pauksktat2011}. The data was collected from
employees of three Dutch preschools, each with two waves of data. Since
there was significant longitudinal incompleteness, the authors of paper
treated each network separately, and carried out a meta analysis for
each wave to test their hypotheses.

They found support for high positions in a recipient's organizational
hierarchy, increasing the likelihood of voice. They also found that both
good social relationship and team co-membership in a dyad increased the
likelihood of voice in that dyad.

The failed to find support for the degree of the recipient or speaker
impacting the likelihood of voice occurring in a dyad. The authors also
concluded that there was not sufficient support to claim that a
speaker's high position in the organizational hierarchy resulted in a
higher likelihood of voice occurring in their dyads.

We were able to replicate published ERGM in only 1 case, however
removing the out-2-star term allowed us to fit a further 4 cases, and
removing the in-2-star term sufficed to allow a model for the final
case. We note that the decay parameters were not specified in the paper,
though we tried possible combinations without being able to match the
published fit. The results were not qualitatively. It seems likely the
effect due to the omission of the 2-star terms, was absorbed by other
terms somewhat.

The authors also did not include an edge parameter in their tables of
their fits. We included an edges parameter, as measure of the baseline
propensity to form edges

We were able to fit the LOLOG model the published ERGM terms in 5 out of
6 networks, where the this were possible the fit to the observed network
was good. For each of these 5 networks we were also able to fit the
LOLOG model with triangle and star terms, which improved the goodness of
fit also.

\subsection{Office Layouts}

As this was a complex example, we showed a detailed fit as our main
example, and omit in this supplement.

\subsection{Disaster Response}

This network is a 20 node directed communication network formed between
various agencies in the search and rescue operation in the aftermath of
a tornado striking a boat on Pomona Lake in Kansas. Because the tornado
destroyed much communication equipment, an important feature of this
network was that the state's highway patrol was the only organisation
having functioning communication equipment. The local sheriff took
control of the operation, and the highway patrol was used for
communication purposes, therefore there are two nodes that are very
highly connected in the observed network. An ERGM was fit in
\cite{Doreian2012} and the data was obtained through
\cite{DisasterData}.

The authors goal in fitting the ERGM was to consider whether local or
global processes lead to the formation of the network. The fit only with
structural ERGM terms and then compared this to a fit using a block
model parameter, it is not specified exactly how this is achieved. The
authors comment that adding the block model parameter yields a superior
fit. The authors did not include nodal covariates in their network.

We were not able to reproduce the ERGM fit stated in the paper. We were
able to fit an ERGM only when omitting the out and mixed star parameters
and including a geometrically weighted in-star parameter. We were able
to fit a LOLOG model using the terms in the published ERGM. With the
omission of nodal covariates these models fit the observed network
poorly.

On including nodal covariates we were able to find an ERGM that fit the
data well, as well as a LOLOG model with the same terms that also fit
the observed network well.

The authors did not provide a detailed interpretation of their fit
mainly using the the ERGM with the block model covariate to argue that
both global and local processes drove the formation of the network.

As the ERGM the LOLOG model with nodal covariates fits well, we argue
that the network and in particular its formation can be explained using
local processes. We also note that the LOLOG with structural terms fits
similarly well to the LOLOG using nodal covariates. This may suggests
that structural social processes are sufficient to explain the network
formation. We note that the LOLOG significant parameter of the in 2
star, and lack of the significant triangles parameter, suggests the
network is driven by a popularity process. This is consistent with the
ERGM fit.

\subsection{Company Boards}

Here we consider the 808 node, undirected networks of interlocking
boards in S\&P 500 companies in the years 2007, 2008, 2009 and 2010. The
nodes in the network are companies, with a tie being present if the
company's board shares members. The network approach using ERGMs to
understand the network, was presented in \cite{Gygax2015}, in particular
to understand tendencies for compensation structure, among companies
that have connected boards.

The authors concluded that once accounting for market size, board
characteristics industry differences, and social structure in the
network, there was a tendency for interlocking board companies to have a
similar proportion of stock compensation in their packages. They also
found the inverse tendency for fixed components of packages, which the
authors interpret as firms that were not connected, tended to
independently anticipate future markets not performing well. Thus in
this period they moved to relying on fixed compensation as a incentive
for executive performance.

The authors supplied the data set without nodal covariates, therefore we
were unable to replicate the reported ERGM fit.

We were able to fit LOLOG models for each of the 4 networks, both with
geometrically weighted degrees and ESP parameters and triangle and star
parameters. We expect a structural fit to fit the data well because the
effect size of the nodal covariates in the data was small. However
fitting the LOLOG with structural terms alone provided a much better fit
than the ERGM using structural terms alone.

\subsection{Swiss Decisions}

The authors in \cite{Fischer2015} investigated directed reputational
trust networks of between 19 and 26 actors in 10 decision making
processes in Switzerland in the 2000s. A node in this networks is an
actor in the decision process, with a tie from actor \(i\) to actor
\(j\) being \(i\) nominating \(j\) as being influential in the decision
making process. The authors argue that aggregating reputational power,
and then proceeding with the analysis, ignores the inherent relational
nature of the data. They argue that to fully model the concept of
reputational power explicitly accounting for the social structure with
ERGM is important.

The authors concluded that ``formal authority, the intensity of
participation in institutional arenas of decision-making processes, and
the centrality in the related collaboration network all have -- albeit
to different extents -- a positive effect on power assessment.'' They
also conclude that actor homophily, preference homophily as well as
collaboration in parallel processes, do not in general impact actors'
assessment of power. These results are regarded as positive by the
authors as an indication that reputational power, is capturing what it
should. They do however note a tendency of collaborators in a single
decision making process to see each other as particularly powerful which
is noted as ``problematic''. Collaboration should improve an actor's
assessment of the other's power but not should not be more likely that
not to increase the perception of power.

We were able to fit the ERGM with the published parameters in 9 out of
10 cases, but the parameter estimates were often inconsistent. Despite
the signs and significance of our estimated parameters not always being
consistent with the published models, our fitted ERGMS in general fitted
the network data well. We were unable to fit the LOLOG model with the
published ERGM terms in 8 out of 10, we suspect this is due to the
correlation between the GWESP and GWDSP (geometrically weighted dyadwise
shared partners) terms. As these were small networks with between 19 and
26 nodes with complex models fit to them we believe the LOLOG models
with triangles and stars were potentially over fitting, achieving a good
fit, yet providing large parameter estimate standard deviations. We
suggest that inference based on such models should be treated with
caution. In general in such small networks it seems that ERGM is often a
preferable model.

\subsection{University Emails}

This is a undirected network of 1133 nodes within a university, with a
connection defined based on a specified frequency of email contact. We
suspect this is not a typical social network, as a connection based on
an email is a very weak social interaction. We note that the authors did
not fit an ERGM using an MLE approach, they selected parameters that
yielded networks that fit on some subjective quantities, the statistical
properties of their analysis are therefore unknown. We do not further
comment on the authors qualitative results, due to the lack of
statistical knowledge of the parameters.

We were able to fit an ERGM with the standard MCMC MLE approach however,
this fit the observed network data very poorly, so we do not discuss it
further. We were able to fit a LOLOG model with triangle and star terms
however we were not able to obtain a good fit to the observed network
and the model had limited interpretability.

In general we do not regard this network as a good example for fitting a
generative social network model based on simple local structures, as the
social connection is very weak, which likely means most of the complex
social structure is not reflected in the data.

\subsection{Elementary School Friendships}

These networks were directed networks of friendships in middle schools
classes on between 22 and 24 children/nodes. The paper that fit this
model was published before MCMC methods for fitting ERGM were widespread
and available. The authors used pseudo likelihood to estimate the
models.

The authors' approach was non standard in the context of modern methods.
They first fit a single network with ``expansiveness'' and
``attractiveness'' parameters for each individual child, essentially a
unique parameter governing the number of friends a child is likely to
nominate as well as the number of times they are likely to be nominated
by other children.

Another model was next fit, regarding the 3 classes as a single model
with no edges between children in different classes. The authors then
fit ERGM with pseudo likelihood with various constraints regarding the
parameters for each of the classes.

The authors concluded there was a tendency towards mutual ties, that did
not differ significantly with gender matching. Attractiveness and
expansiveness interpretation was presented on an individual child basis,
with the authors observing improved fit with the inclusion of these
parameters.

As this was a non standard modelling approach we did not recreate the
published ERGM fits directly. We were able to fit the ERGM model with
MCMC MLE methods, with GWESP and GWDEG terms for the grade 4 and 5
models, but needed to omit the GWESP terms to be able to fit the grade 3
network. All models showed strong homophily on grade, with the GWESP
term significant and positive and the GWDEG terms not significant for
grades 4 and 5. The simpler grade 3 model had significant and negative
terms for GWDEG terms suggesting that the network was not driven by
super friendship nominators or nomination receivers. These models fitted
the observed network data well.

We were not able to fit LOLOG models to these networks using the
published ERGM terms, however using triangle and star terms we were able
to achieve a better fit with the LOLOG model. However the LOLOG model
parameters had large standard errors in line with our experiences with
very small approximately 20 node networks, so for the grade 4 and 5
networks the ERGM model with modern terms was preferable. As we were
unable to fit an ERGM to the grade 3 model with the GWESP term and the
ERGM with GWDEG terms did not fit this network well, so we suggest the
LOLOG model was more suitable for modelling the grade 3 network.

\subsection{Online Links}

These networks are directed and undirected networks of websites with
hyperlinks and similar ``framing'' of issues respectively. The hyperlink
network had 158 websites/nodes whereas the framing network had 150
websites/nodes.

The authors noted significant homophily in the three social movement
categories they defined, so that sites were more likely to be linked and
frame issues similarly if they belonged to the same social movement.
Though this effect was stronger in the hyperlink network. They note that
all their structural coefficients in the hyperlink network were positive
and significant, informing them of what they refer to as ``informal
linking'' in the hyperlink network, that is a network structure
consistent with ``social interactions''. They note that the framing
network is denser with greater centralisation than the link network,
suggesting there is significant unconscious connection between websites
and a tendency for decentralisation within social movements.

We were able to recreate the published fits in both cases however found
that the models did not fit the observed networks well. We found the
recreated ERGM for the hyperlink network in particular fit very poorly.
We were able to fit the LOLOG models with the ERGM terms but both models
had similarly poor fit on the observed data.

We were able to fit LOLOG models using triangle and star terms to
achieve a good fit to the observed data, and therefore recommend LOLOG
as a better model for explaining these networks.

\section{Links to publicly available data}

Table \ref{tab:public_data} provides hyperlinks to the publicly
available datasets used in our ensemble.

\begin{longtable}[t]{ll}
\caption{\label{tab:unnamed-chunk-1}\label{tab:public_data} Links to publicly available datasets}\\
\toprule
Network & Links\\
\midrule
\rowcolor{gray!6}  Add Health & addhealth.cpc.unc.edu/\\
Elementary School & moreno.ss.uci.edu/data.html\#children\\
\rowcolor{gray!6}  Florentine Families & sites.google.com/site/ucinetsoftware/datasets/padgettflorentinefamilies\\
Kapferer's Tailors & sites.google.com/site/ucinetsoftware/datasets/kapferertailorshop\\
\rowcolor{gray!6}  Natural Disasters & vlado.fmf.uni-lj.si/pub/networks/data/GBM/kansas.htm\\
\addlinespace
German Schoolboys & github.com/gephi/gephi/wiki/Datasets\\
\rowcolor{gray!6}  Company Boards & corp.boardex.com\\
\bottomrule
\end{longtable}

\bibliographystyle{chicago}
\bibliography{bib}

\end{document}
